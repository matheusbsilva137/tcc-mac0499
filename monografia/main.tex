% Arquivo LaTeX de exemplo de artigo
%
% Criação: Jesús P. Mena-Chalco
% Revisão: Fabio Kon e Paulo Feofiloff
% Adaptação para UTF8, biblatex e outras melhorias: Nelson Lago
%
% Except where otherwise indicated, these files are distributed under
% the MIT Licence. The example text, which includes the tutorial and
% examples as well as the explanatory comments in the source, are
% available under the Creative Commons Attribution International
% Licence, v4.0 (CC-BY 4.0) - https://creativecommons.org/licenses/by/4.0/


%%%%%%%%%%%%%%%%%%%%%%%%%%%%%%%%%%%%%%%%%%%%%%%%%%%%%%%%%%%%%%%%%%%%%%%%%%%%%%%%
%%%%%%%%%%%%%%%%%%%%%%%%%%%%%%% PREÂMBULO LaTeX %%%%%%%%%%%%%%%%%%%%%%%%%%%%%%%%
%%%%%%%%%%%%%%%%%%%%%%%%%%%%%%%%%%%%%%%%%%%%%%%%%%%%%%%%%%%%%%%%%%%%%%%%%%%%%%%%

% Se você está escrevendo um artigo para algum periódico que fornece um
% modelo LaTeX, é melhor usá-lo (e, se necessário, você pode copiar coisas
% úteis deste modelo). Se não houver um modelo preexistente, você pode tentar
% adaptar este para o formato esperado, mas essa não é uma tarefa trivial.
%
% A opção twoside (frente-e-verso) significa que a aparência das páginas pares
% e ímpares pode ser diferente. Por exemplo, as margens podem ser diferentes ou
% os números de página podem aparecer à direita ou à esquerda alternadamente.
% Mas nada impede que você crie um documento "só frente" e, ao imprimir, faça
% a impressão frente-e-verso.
%
% Aqui também definimos a língua padrão do documento
% (a última da lista) e línguas adicionais.
%\documentclass[12pt,twoside,brazilian,english]{article}
\documentclass[12pt,twoside,english,brazilian]{article}

% Ao invés de definir o tamanho das margens, vamos definir os tamanhos do
% texto, do cabeçalho e do rodapé, e deixamos a package geometry calcular
% o tamanho das margens em função do tamanho do papel. Assim, obtemos o
% mesmo resultado impresso, mas com margens diferentes, se o tamanho do
% papel for diferente.
\usepackage[a4paper]{geometry}

\geometry{
  textwidth=152mm,
  hmarginratio=12:17, % 24:34 -> com papel A4, 24mm + 152mm + 34mm = 210mm
  textheight=237mm,
  vmarginratio=8:7, % 32:28 -> com papel A4, 32mm + 237mm + 28mm = 297mm
  headsep=11mm, % distância entre a base do cabeçalho e o texto
  headheight=21mm, % qualquer medida grande o suficiente, p.ex., top - headsep
  footskip=10mm,
  marginpar=20mm,
  marginparsep=5mm,
}

% Vários pacotes e opções de configuração genéricos; para personalizar o
% resultado, modifique estes arquivos.
\input{extras/basics}
\input{extras/languages}
\input{extras/fonts}
\input{extras/floats}
\input{extras/imeusp-formatting}
\input{extras/index}
\input{extras/bibconfig}
\input{extras/hyperlinks}
%\nocolorlinks % para impressão em P&B
\input{extras/source-code}
\input{extras/utils}

% Diretórios onde estão as figuras; com isso, não é preciso colocar o caminho
% completo em \includegraphics (e nem a extensão).
\graphicspath{{figuras/},{logos/}}

% Comandos rápidos para mudar de língua:
% \en -> muda para o inglês
% \br -> muda para o português
% \texten{blah} -> o texto "blah" é em inglês
% \textbr{blah} -> o texto "blah" é em português
\babeltags{br = brazilian, en = english}

% Espaçamento simples
\singlespacing

% Bibliografia
\usepackage[
  style=extras/plainnat-ime, % variante de autor-data, similar a plainnat
  %style=alphabetic, % similar a alpha
  %style=numeric, % comum em artigos
  %style=authoryear-comp, % autor-data "padrão" do biblatex
  %style=apa, % variante de autor-data, muito usado
  %style=abnt,
]{biblatex}


%%%%%%%%%%%%%%%%%%%%%%%%%%%%%%%%%%%%%%%%%%%%%%%%%%%%%%%%%%%%%%%%%%%%%%%%%%%%%%%%
%%%%%%%%%%%%%%%%%%%%%%%%%%%%%%%%%%% METADADOS %%%%%%%%%%%%%%%%%%%%%%%%%%%%%%%%%%
%%%%%%%%%%%%%%%%%%%%%%%%%%%%%%%%%%%%%%%%%%%%%%%%%%%%%%%%%%%%%%%%%%%%%%%%%%%%%%%%

% O arquivo com os dados bibliográficos para biblatex; você pode usar
% este comando mais de uma vez para acrescentar múltiplos arquivos
\addbibresource{bibliografia.bib}

% Este comando permite acrescentar itens à lista de referências sem incluir
% uma referência de fato no texto (pode ser usado em qualquer lugar do texto)
%\nocite{bronevetsky02,schmidt03:MSc, FSF:GNU-GPL, CORBA:spec, MenaChalco08}
% Com este comando, todos os itens do arquivo .bib são incluídos na lista
% de referências
%\nocite{*}

% Estes comandos definem o título e autoria do trabalho e devem sempre ser
% definidos, pois além de serem utilizados para criar a capa (com o comando
% \maketitle), também são armazenados nos metadados do PDF. O estilo padrão
% de diversos periódicos exige também outros dados, como email, filiação etc.
\title{Estudo do efeito de variações de Bloom filters no desempenho de consultas no banco de dados Apache Cassandra}
\author{
  Matheus Barbosa Silva\thanks{Instituto de Matemática e Estatística da Universidade de São Paulo}
}

% O pacote hyperref armazena alguns metadados no PDF gerado (em particular,
% o conteúdo de "\title" e "\author"). Também é possível armazenar outros
% dados, como uma lista de palavras-chave ou o resumo.
\hypersetup{
  pdfkeywords={LaTeX, artigo},
  pdfsubject={Uma variante do arquivo tese.tex usando a classe article.},
}

% É possível definir como determinadas palavras podem (ou não) ser
% hifenizadas; no entanto, a hifenização automática geralmente funciona bem
\babelhyphenation{documentclass latexmk soft-ware} % todas as línguas
\babelhyphenation[brazilian]{Fu-la-no}
\babelhyphenation[english]{what-ever}

% Por padrão, article inclui a data atual; com este comando, você pode
% definir uma data específica, inserir algum outro texto ou, deixando o
% conteúdo em branco, removê-la.
\date{}


%%%%%%%%%%%%%%%%%%%%%%%%%%%%%%%%%%%%%%%%%%%%%%%%%%%%%%%%%%%%%%%%%%%%%%%%%%%%%%%%
%%%%%%%%%%%%%%%%%%%%%%%%%%%%%%% INÍCIO DO ARTIGO %%%%%%%%%%%%%%%%%%%%%%%%%%%%%%%
%%%%%%%%%%%%%%%%%%%%%%%%%%%%%%%%%%%%%%%%%%%%%%%%%%%%%%%%%%%%%%%%%%%%%%%%%%%%%%%%

\begin{document}

% Gera a "capa" do artigo (geralmente, título, autor etc. sem que haja uma
% quebra de página para o restante do conteúdo)
\maketitle

\begin{abstract}
  Uma variante do arquivo \texttt{tese.tex} usando a classe \textsf{article}.
\end{abstract}

\section{Introdução}

Se você precisa criar um texto relativamente curto, como um artigo ou
um trabalho de disciplina, este modelo pode servir como base. Observe,
no entanto, que periódicos em geral nas áreas de matemática e computação
costumam ter seus próprios modelos \LaTeX{} (como é o caso da
SBC\nocite{sbctemplate});
nesses casos, é melhor utilizá-los e apenas consultar este modelo para
verificar como usar algum recurso específico. Fique atento: alguns modelos
antigos ou de periódicos internacionais podem usar \textsf{latin1} ao
invés de \textsf{utf8} ou mesmo não ter configuração pré-definida para
caracteres acentuados. Além disso, eles muito frequentemente utilizam
bibtex ao invés de biblatex para a geração automática da bibliografia.

\printbibliography[
  title=\refname\label{bibliografia}, % "Referências", recomendado pela ABNT
  %title=\bibname\label{bibliografia}, % "Bibliografia"
]

\newpage

\section{\textit{Approximate set membership}}

A representação de \textbf{conjuntos} de elementos por meio de estruturas de dados requer que essas estruturas sejam capazes de indicar não só as informações armazenadas, mas também responder quais elementos estão no conjunto. Nesse contexto, a verificação de que um dado elemento $x$ é membro de um conjunto $S$ (isto é, $x \in S$) é chamada de \textit{membership testing} (teste de membresia).

A construção de soluções para o problema de verificar se um elemento pertence a um conjunto pode ser baseada em uma das seguintes variantes do problema: \textbf{estática} ou \textbf{dinâmica}. Essa classificação é feita de acordo com as informações dadas no momento de execução de buscas de elementos no conjunto. Na variante estática do problema, assume-se que o conjunto $S$ tem um tamanho fixo, logo todos os elementos do conjunto são expressos antes das consultas. Já na variante dinâmica, inserções e consultas podem estar intercaladas.

Para ambas as variantes do problema, a construção de uma estratégia eficiente que permita obter uma \textbf{resposta determinística} para os \textit{membership tests} pode não ser imediata. Uma solução intuitiva, mas que demanda o consumo linear de espaço é a comparação do elemento a ser verificado com cada um dos elementos do conjunto. No entanto, em alguns cenários, a escalabilidade é um requisito fundamental, portanto é desejável que se evite o consumo linear de espaço.

Assim, propõe-se o \textit{approximate set membership} (teste de membresia aproximado) como uma \textbf{solução probabilística} que permite responder consultas sobre um conjunto de forma aproximada, com o custo de resultados falsos positivos. Um resultado falso positivo ocorre com probabilidade $\epsilon$, logo $0 < \epsilon < 1$.

\subsection{Implementação de Dicionário Completo}

Dicionários completos são estruturas de dados que apresentam uma entrada para cada elemento de um conjunto (isto é, há enumeração completa dos elementos).

\subsubsection{Interface}

Um dicionário completo deve implementar os seguintes comandos:
\begin{itemize}
    \item \texttt{INSERT(x)}: insere um dado elemento $x$ no conjunto $S$;
    \item \texttt{QUERY(x)}: verifica se um dado elemento $x$ é membro do conjunto $S$ (\textit{membership testing});
\end{itemize}

\subsubsection{\textit{Hash table}}

\textit{Hash table} é uma implementação possível para a estrutura de dicionário completo. Essa estrutura mantém um array de \textit{buckets} que comportam zero ou mais elementos do conjunto $S$. Também escolhe-se uma função de \textit{hashing} $h$, $h: \mathcal{U} \xrightarrow{} \{0,1,...,m\}$ que mapeia cada elemento do conjunto a um \textit{bucket} indexado por um número no intervalo $[0,m]$, em que $m < |\mathcal{U}|$. Logo, no caso de colisão de \textit{hashing} (onde vários elementos do conjunto são mapeados para um \textit{bucket} de mesmo índice) deve existir um \textbf{esquema de resolução de colisão}.

Um esquema de resolução de colisão deve permitir que vários elementos do conjunto mapeados para o mesmo \textit{bucket} possam ser identificados individualmente. Um possível esquema de resolução de colisão é a construção de uma lista ligada que contenha todos os elementos mapeados para um mesmo \textit{bucket}.

Uma implementação de dicionário completo com \textit{hash table} deve aplicar os procedimentos seguintes para cada comando:
\begin{itemize}
    \item \texttt{INSERT(x)}: usa-se a função de hashing $h$ para obter um índice $i = h(x)$, $0 \leq i < m$. O elemento é, então, inserido na estrutura no \textit{bucket} de índice $i$, que emprega o esquema de resolução de colisões, se necessário;
    \item \texttt{QUERY(x)}: usa-se a função de hashing $h$ para obter um índice $i = h(x)$, $0 \leq i < m$. Então, busca-se o elemento no \textit{bucket} de índice $i$. Se o elemento é encontrado, retorna que $x \in S$, senão retorna que $x \notin S$.
\end{itemize}

Seja $u = |\mathcal{U}|$ o tamanho do conjunto universo e $n=|S|$ o tamanho do conjunto contido em $U$, logo, há $\binom{u}{n}$ conjuntos $S$ distintos que podem ser representados pela estrutura. Portanto, a memória necessária, em bits, para representar computacionalmente todos os possíveis conjuntos $S$ é limitada superiormente por $\log_2 \binom{u}{n} = n \log_2 \frac{u}{n} + \theta(n)$. \footnote{Pretendo destrinchar melhor essa passagem}

Essa implementação permite que consultas sejam realizadas com tempo esperado $O(1)$ ou, no pior caso, em tempo $\theta(n)$. Já inserções são sempre realizadas em tempo $O(1)$ desde que se considere um esquema de resolução de conflitos também $O(1)$ (como a implementação de uma lista ligada, que permite a inserção de novos elementos em tempo $\theta(1)$). \footnote{citar fonte ou demonstrar} No entanto, elementos com comprimentos grandes podem requerer vários acessos ao disco e causar perda de desempenho, logo é desejável minimizar a quantidade de acessos ao disco, de modo que o consumo real de tempo aproxime-se do esperado.

\subsubsection{\textit{Hash Compaction}}

A implementação de dicionários completos com \textit{hash compaction}, introduzida por Wolper e Leroy, viabiliza uma \textbf{relação linear} entre o número de elementos do conjunto e o espaço consumido pela estrutura. Essa estrutura armazena os elementos do conjunto em \textbf{blocos de tamanho fixo}, assim todo elemento inserido na estrutura consome o mesmo espaço (em bits). Para isso, usa-se uma nova função de hashing $H$, $H: \mathcal{U} \mapsto \{0, 1\}^l$, onde $l$ é o comprimento dos blocos na estrutura.

Todo elemento é comprimido em $l$ bits pela função de hashing $H$ antes de ser inserido na estrutura. Implementações com \textit{hash compaction} devem aplicar os seguintes procedimentos para cada comando:

\begin{itemize}
    \item \texttt{INSERT(x)}: aplica-se o procedimento de inserção usado na implementação com \textit{hash table} para incluir uma representação comprimida de $x$ na estrutura. Seja $x' = H(x)$ a representação comprimida de $x$ e $i = h(x')$, $0 \leq i < m$, o índice do \textit{bucket} do elemento na estrutura. A representação comprimida $x'$ é inserida na estrutura no bucket de índice $i$, que emprega o esquema de resolução de colisões, se necessário;
    \item \texttt{QUERY(x)}: aplica-se a função de hashing $H$ para obter $x' = H(x)$, uma representação comprimida de $x$. Então, usa-se a função de hashing $h$ para obter um índice $i = h(x')$, $0 \leq i < m$. Se $x'$ é encontrado no \textit{bucket} de índice $i$, retorna que $x \in S$, senão retorna que $x \notin S$.
\end{itemize}

\end{document}
